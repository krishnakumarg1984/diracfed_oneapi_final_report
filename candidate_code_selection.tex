% chktex-file 24
% chktex-file 2
% chktex-file 8
% chktex-file 44

\documentclass[main]{subfiles}

\begin{document}

\section{Candidate Code Selection}\label{sec:codesel}
Since the project funding comes from UKRI the codes proposed for porting can cover the whole remit of UKRI research.
Given the limited time and scope of the project we preferred codes that already had a GPU port and that already had tests.
The preference for tests is because we wanted to have confidence that the ported code was correct (or, at least bug compatible with the original code).
For ease of replication, it is preferable to use codes with permissive licenses.

dGpoly3D is a code under development that consists of various compute kernels.It does not yet have a public release.
After the initial short-listing of codes, we decided to add AREPO to the project to investigate the experience of porting a CPU-only code.

\begin{table}[!htbp]
	\begin{tabular}{@{}llllSlc@{}}
		\toprule
		\thead{Name} & \thead{Domain}     & \thead{Language} & \thead{Directives} & {\thead{LoC}}      & \thead{Licence} & \thead{Code                                         \\ Repository}                                        \\
		\midrule
		OpenQCD      & Particle physics   & C                &                    & \qty{110}{\kilo{}} & GPL 2.0         & \cite{fastsum_collaboration_openqcd-fastsum_nodate} \\
		OpenMM       & Molecular dynamics & C++              & CUDA/OpenCL        & \qty{650}{\kilo{}} & Unknown         & \cite{noauthor_openmm_nodate}                       \\
		HemeLB       & Medical physics    & C++              & CUDA               & \qty{85}{\kilo{}}  & LGPL 3.0        & \cite{hemelb_authors_hemelb_nodate}                 \\
		dGpoly3D     &                    & C++              & CUDA               & {---}              & GPL 3.0         & \cite{dgpoly3d_nodate}                              \\
		AREPO        & Astrophysics       & C                & OpenMP             & \qty{67}{\kilo{}}  & GPL 3.0         & \cite{weinberger_arepo_2020}                        \\
		\bottomrule
	\end{tabular}
	\caption{Candidate Codes}
	\label{tab:candidate codes}
\end{table}

\end{document}
