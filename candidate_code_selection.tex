% chktex-file 24
% chktex-file 2
% chktex-file 8
% chktex-file 44

\documentclass[main]{subfiles}

\begin{document}

\section{Candidate Code Selection}\label{sec:codesel}
Since the project funding comes from UKRI the codes proposed for porting can cover the whole remit of UKRI research.
Given the limited time and scope of the project we preferred codes that already had a GPU port and that already had tests.
The preference for tests is because we wanted to have confidence that the ported code was correct (or, at least bug compatible with the original code).
For ease of replication it would be preferable that the code was openly licensed.

dGpoly3D is a code under development and not yet publicly release that consists of various compute kernels.

After the initial selection an opportunity arose with come other work being done on AREPO at Cambridge to add it to the project to investigate the experience of porting a CPU only code.

\begin{table}[!htbp]
	\begin{tabular}{@{}llllllr@{}}
		\toprule
		\thead{Name} & \thead{Domain}    & \thead{Language} & \thead{Directives} & \thead{LoC} & \thead{Licence} & \thead{Repo}                                        \\
		\midrule
		OpenQCD      & Particle pyhsics  & C                &                    & 110k        & GPL 2.0         & \cite{fastsum_collaboration_openqcd-fastsum_nodate} \\
		OpenMM       & Molecular dyamics & C++              & CUDA/OpenCL        & 650k        & Unknown         & \cite{noauthor_openmm_nodate}                       \\
		HemeLB       & Medical physics   & C++              & CUDA               & 85k         & LGPL 3.0        & \cite{hemelb_authors_hemelb_nodate}                 \\
		dGpoly3D     &                   & C++              & CUDA               & -           & GPL 3.0         & \cite{dgpoly3d_nodate}                                                   \\
		AREPO        & Astrophysics      & C                & OpenMP             & 67k         & GPL 3.0         & \cite{weinberger_arepo_2020}                        \\
		\bottomrule
	\end{tabular}
	\caption{Candidate Codes}
	\label{tab:candidate codes}
\end{table}

\end{document}
