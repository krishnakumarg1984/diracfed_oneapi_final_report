% chktex-file 13

\documentclass[main]{subfiles}

\begin{document}

\section{Introduction}\label{sec:intro}
%\begin{itemize}
%    \item Need for performance
%    \item NVIDA hw CUDA sw \textit{de facto} standard
%    \item hw vendor diversity NSF pre-exascle deployments
%    \item EuroHPC JU plans
%    \item science code lifetimes
%    \item implies SYCL
%\end{itemize}

The DiRAC~\cite{dirac_distributed_nodate} facility provides distributed High Performance Computing (HPC) services to the STFC theory community.
HPC-based modelling is an essential tool for the exploitation and interpretation of observational and experimental data generated by astronomy and particle physics facilities support by STFC as this technology allows scientists to test their theories and run simulations from the data gathered in experiments.

Up to and including the current (early 2022) generation of HPC, the market for GPU accelerators has been dominated by NVIDIA hardware.
By extension, the CUDA software framework that has been developed to most efficiently exploit this hardware has become the first-choice programming environment for many use cases.
However, the current progression towards exascale computing has seen new hardware alternatives being deployed on next-generation HPC.
Notable examples include:

\begin{itemize}
	\item LUMI (Finland) --- AMD GPUs --- Will be fastest computer in Europe on completion (mid-late 2022)
	\item Aurora (USA) --- Intel GPUs
	\item Frontier (USA) --- AMD GPUs
	\item El Capitan (USA) --- AMD GPUs
	\item SuperMUC-NG Phase 2 (Germany) --- Intel GPUs
\end{itemize}

\end{document}
