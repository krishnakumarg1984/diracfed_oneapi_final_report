%!TEX root = ../main.tex
% vim:nospell ft=tex

\renewcommand{\arraystretch}{1.3}

% \newcolumntype{P}[1]{>{\RaggedRight\hspace{0pt}}p{#1}}
% % \newcolumntype{R}[1]{>{\raggedleft\let\newline\\\arraybackslash\hspace{0pt}}m{#1}}
% \newcolumntype{C}[1]{>{\centering\arraybackslash}p{#1}}

% \makeatletter
%
% \newcommand*{\@rowstyle}{}
%
% \newcommand*{\rowstyle}[1]{% sets the style of the next row
%     \gdef\@rowstyle{#1}%
%     \@rowstyle\ignorespaces%
% }
%
% \newcolumntype{=}{% resets the row style
%     >{\gdef\@rowstyle{}}%
% }
%
% \newcolumntype{+}{% adds the current row style to the next column
%     >{\@rowstyle}%
% }
%
% \makeatother



% \makeatletter
% \newlength{\qrr@dimen@}
% \expandafter\pretocmd\csname tabular*\endcsname{\setlength{\qrr@dimen@}{#1}}{}{}
% \newcommand*{\Rowcolor}[2][\tabcolsep]{%
%     \ifx\relax#1\relax\else
%         \kern-\the\dimexpr#1\relax
%     \fi
%     \makebox[0pt][l]{%
%         \fboxsep=0pt
%         \colorbox{#2}{%
%             \strut\kern\qrr@dimen@
%         }%
%     }%
%     \ifx\relax#1\relax\else
%         \kern\the\dimexpr#1\relax
%     \fi
%     \ignorespaces
% }
% \makeatother
