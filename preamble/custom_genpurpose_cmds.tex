%!TEX root = ../main.tex
% vim:nospell ft=tex
% chktex-file 1

% these custom commands  are general purpose definitions that are  suitable in a
% typical scientific	document. Some	of them  are pure  latex while	others use
% external packages

\newcommand\blankpage{
	\null
	\thispagestyle{empty}
	\addtocounter{page}{-1}
	\newpage
}

%---------- for text and other typographical elements ----------%
\newcommand{\eg}{\textit{e}.\textit{g}.}
\newcommand{\etal}{\textit{et~al}.}
\newcommand{\ie}{\textit{i}.\textit{e}.,}
\newcommand{\viz}{\textit{viz}. }

\setlength\parskip{0.75\baselineskip plus0.1\baselineskip  minus0.1\baselineskip}

% https://tex.stackexchange.com/questions/23487/how-can-i-get-roman-numerals-in-text
% \makeatletter
% \newcommand*{\romanletter}[1]{\expandafter\@slowromancap\romannumeral #1@}
% \makeatother

% improved handling of sectioning commands with titlesec package
% \setcounter{secnumdepth}{3} % organisational level that receives a numbers
% \setcounter{tocdepth}{3}		% print table of contents for level 3

% \setlength{\columnsep}{20pt} % space between columns in two column mode; default 10pt quite narrow

\renewcommand{\footnoterule}{%
	\kern -3pt
	\hrule width 0.25\textwidth height 0.5pt
	\kern 2pt
}

% % https://tex.stackexchange.com/questions/29916/how-to-place-a-footnote-inside-a-float-environment
% \newcommand{\mpfootnotes}[1][1]{
%			\renewcommand{\thempfootnote}{\thefootnote}
%			\addtocounter{footnote}{-#1}
% \renewcommand{\footnote}{\stepcounter{footnote}\footnotetext[\value{footnote}]}}
