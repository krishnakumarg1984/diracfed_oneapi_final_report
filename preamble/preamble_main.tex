%!TEX root = ../main.tex
% vim:nospell

% \directlua{pdf.setminorversion(7)}  % chktex 36
% \RequirePackage[debrief]{silence}
\RequirePackage[l2tabu, orthodox]{nag}

% \begingroup\newif\ifmy % chktex 1
% \IfFileExists{./\jobname.xmpdata}{}{\mytrue} % https://tex.stackexchange.com/questions/98203/can-i-test-if-a-file-exists
% \ifmy\begin{filecontents}{\jobname.xmpdata}
%         %!TEX root = ../main.tex
% vim:nospell

\Author{Christopher Edsall \and Kacper Kornet \and Stefanie Reuter \and Thomas Kappas \and Tuomas Koskela \and Krishnakumar Gopalakrishnan \and Alexander Wade \and Ioannis Zacharoudiou \and Jon McCullough \and Shunzhou Wan \and James Richings}  % chktex 1
\Copyright{\textcopyright{} The authors. All rights reserved.}
\Creator{LuaTeX}
\Subject{Heterogenous Computing \sep SyCL \sep Portable Software \sep OpenCL \sep GPU \sep DiRAC \sep FPGA \sep High Performance Computing \sep Intel OneAPI \sep Software Development \sep Scientific Research \sep Numerical Software \sep Parallel Programming \sep Scientific Computing} % chktex 1
\Keywords{Heterogenous Computing \sep SyCL \sep Portable Software \sep OpenCL \sep GPU \sep DiRAC \sep FPGA \sep High Performance Computing \sep Intel OneAPI \sep Software Development \sep Scientific Research \sep Numerical Software \sep Parallel Programming \sep Scientific Computing} % chktex 1
\Producer{LuaTeX}
\PublicationType{Report}
\Publisher{Distributed Research Utilising Advanced Computing (DiRAC), UK}
\Title{DiRAC Technical Report --- DiRAC Federation Project WP 3.2.1}

%     \end{filecontents}
% \fi\endgroup

% These options are passed to the relevant packages when we load them later explicitly with \usepackage (or when some other package loads them implicity as their dependencies).
% Using PassOptionsToPackage is the idiomatic way to pass options to latex packages. This is a superior alternative to \usepackage[option]{packagename}  % see https://texfaq.org/FAQ-optionclash

%% Uncomment the following package options as needed (packages are not loaded though)
% \PassOptionsToPackage{title, titletoc}{appendix}
% \PassOptionsToPackage{nameinlink}{cleveref}
% \PassOptionsToPackage{inline}{enumitem}
% \PassOptionsToPackage{bottom}{footmisc}
% \PassOptionsToPackage{nomain, acronym, symbols, stylemods={tree}, numberedsection=nameref}{glossaries-extra}
% \PassOptionsToPackage{nomain, acronym, symbols, record, nonumberlist, stylemods={tree},style=treegroup,postdot}{glossaries-extra}
% \PassOptionsToPackage{section}{placeins}

% Adjust options as per requirements
\PassOptionsToPackage{british}{babel}
\PassOptionsToPackage{backend=biber, style=trad-plain, sorting=none, citestyle=numeric-comp, maxbibnames=50, url=true, doi=true, eprint=false}{biblatex}
\PassOptionsToPackage{depth=3, open=true, openlevel=0, numbered=true}{bookmark}
\PassOptionsToPackage{margin=10pt, font=small, labelfont={bf}, labelsep=quad}{caption}
\PassOptionsToPackage{draft}{fixme} % ok to have fixme and todonotes (see a few lines below). No packages are loaded here. This merely specifies what options to pass if packages get loaded later on.
% \PassOptionsToPackage{no-math, quiet}{fontspec}
\PassOptionsToPackage{paper=a4paper, hmarginratio=1:1, vmarginratio=1:1, scale=0.765}{geometry}
% \PassOptionsToPackage{libertinus}{fontsetup}
\PassOptionsToPackage{all}{hypcap}
\PassOptionsToPackage{pdfencoding=auto,psdextra}{hyperref}
\PassOptionsToPackage{hyphenation, lastparline, nosingleletter}{impnattypo}
\PassOptionsToPackage{frac, vfrac, multskip}{mathfixs}
\PassOptionsToPackage{final, activate={true, nocompatibility}, factor=1100, stretch=10, shrink=10, babel=true,tracking=true,babel=true}{microtype}
\PassOptionsToPackage{defaultlines=2, all}{nowidow}
\PassOptionsToPackage{british}{selnolig}
\PassOptionsToPackage{separate-uncertainty = true, multi-part-units=single, detect-all}{siunitx}
\PassOptionsToPackage{textsize=scriptsize,textwidth=2.25cm}{todonotes}  % ok to have fixme (see a few lines above) and todonotes. No packages are loaded here. This merely specifies what options to pass if packages get loaded later on.
\PassOptionsToPackage{normalem}{ulem}
\PassOptionsToPackage{warnings-off={mathtools-colon,mathtools-overbracket,mathtools-underbracket}}{unicode-math}
\PassOptionsToPackage{table}{xcolor}

\documentclass[12pt,a4paper]{article}
\usepackage[utf8]{inputenc}
\usepackage[T1]{fontenc}

%%%%%%%%%% list of packages to be loaded %%%%%%%%%%

%!TEX root = ../main.tex
% vim:nospell

% The 'microtype' package is not loaded in this file, since it must be loaded after the 'babel' package (which itself is to be loaded after the 'fontspec' package)

\usepackage{authblk}

\usepackage{babel}         % with the 'british' option, we get OUP hyphenation material for free
\usepackage{biblatex}      % strong dependence on csquotes

\usepackage{checkend}

\usepackage{fancyhdr}
\usepackage{filemod}
% \usepackage{fixme}

\usepackage{geometry}

\usepackage{labelschanged}

\usepackage{setspace}      % Used to define line spacing

% \usepackage{siunitx}
% %!TEX root = ../main.tex
% vim:nospell ft=tex

\sisetup{
	locale = UK,
	per-mode = reciprocal-positive-first,
	group-minimum-digits = 9,
}
\sisetup{range-phrase=--}
\sisetup{range-units=single}

% \robustify\tnote
% \newrobustcmd{\Tnote}[1]{\textsuperscript {\TPTtagStyle {#1}}} % needs etoolbox which is most probably loaded by other packages such as glossaries* https://tex.stackexchange.com/questions/449908/align-columns-at-decimal-point-using-siunitx-wherein-each-cell-contains-superscr/449992?noredirect=1#comment1131112_449992


\usepackage{todonotes}     % whoa! loads array,color,colortbl,pdfcolmk,xcolor,tikz,pgf,pgffor,pgfrcs,pgfcomp-*,graphicx,keyval,graphics,trig,pgfsys,xcolor,pgfbasedecorations,pgfbasesnakes,


% ------- other unused general packages (but potentially useful in a typical article/report) ----------
% \usepackage{addlines}
% \usepackage{blindtext}
% \usepackage{cmdtrack}
% \usepackage{etoolbox}   % not really required if using glossaries package, since this package then gets loaded automatically
% \usepackage{fnpct}
% \usepackage{footnote}
% \usepackage{fullpage}
% \usepackage{layouts}    % helpful for computing textwidth, textheight etc
% \usepackage{lscape}
% \usepackage{luabibentry}
% \usepackage{makebox}
% \usepackage{needspace}
% \usepackage{nolbreaks}
% \usepackage{pdflscape}
% \usepackage{soulutf8}
% \usepackage{subfiles}
% \usepackage{ulem}
% \usepackage{varwidth}
% \usepackage[table]{xcolor} % loaded by pdfx package (if pdfx is loaded). Note that the user cannot explicitly load the colortbl package either before or after when using xcolor. xcolor is also loaded by the todonotes package
        % general packages for a typical technical report; loads the biblatex package
%!TEX root = ../main.tex
% vim:nospell

\usepackage{booktabs}
% \usepackage{caption}     % loaded by 'subcaption'. For improved layout of figure captions with extra margin, smaller font than text etc.
% \usepackage{float}      % Loading order is important here; https://tex.stackexchange.com/questions/435529/correct-loading-order-of-package-newfloat-along-with-hyperref-and-algorithmic-pa/435597#435597

\usepackage{graphicx}     % 'graphicx' must be loaded before 'fontspec'
%!TEX root = ../main.tex
% vim:nospell ft=tex

\graphicspath{{./images}}

% \DeclareGraphicsExtensions{.pdf, .png, .jpg, .jpeg} % GIF doesn't work??
% \setkeys{Gin}{width=0.75\textwidth} % default width of graphics

% https://tex.stackexchange.com/questions/1072/which-graphics-formats-can-be-included-in-documents-processed-by-latex-or-pdflat
% prepend pdf before png
% \ifpdf
%     \makeatletter
%     \let\orig@Gin@extensions\Gin@extensions
%     \def\Gin@extensions{.pdf,\orig@Gin@extensions} %prepend .pdf before .png
%     \makeatother
% \fi


\usepackage{floatrow}
%!TEX root = ../main.tex
% vim:nospell ft=tex

% https://tex.stackexchange.com/a/27100
% \DeclareFloatFont{tiny}{\tiny}% "scriptsize" is defined by floatrow, "tiny" not
\floatsetup[table]{font=small}


\usepackage{makecell}
%!TEX root = ../main.tex
% vim:nospell ft=tex

% https://tex.stackexchange.com/a/351118
\renewcommand\theadfont{\bfseries}


\usepackage{multirow}
\usepackage{subfig}       % better to use subcaption instead of subfig
% \usepackage{subcaption}  % already loads 'caption'


% ---------- Other unused float-related packages  ------------------
% \usepackage{chkfloat}
% \usepackage{flafter}
% \usepackage{longtable}
% \usepackage{makecell}
% \usepackage{multicol}
% \usepackage{placeins}       % Defines a \FloatBarrier command
% \usepackage{rotating}       % defines a sidewaystable environment
% \usepackage{tablefootnote}
% \usepackage{tabularx}
  % table/images/caption related packages (latex float handling)
% %!TEX root = ../main.tex
% vim:nospell

% \usepackage{amsmath}
% \usepackage{amsfonts}
% \usepackage{amssymb}

% \usepackage{diffcoeff}

% \usepackage{lualatex-math}  % will only work with the LuaTeX engine

% \usepackage[all,warning]{onlyamsmath} % if using Tikz, please include the calc and babel libraries (known incompatibilities)

% \usepackage{mathtools}
% \usepackage{mathfixs}

% \usepackage{xfrac}

%!TEX root = ../main.tex
% vim:nospell ft=tex
% chktex-file 1

%---------- inline/display math macros ----------%
% \newcommand*\mean[1]{\overline{#1}}

% \DeclarePairedDelimiter\abs{\lvert}{\rvert}
% \DeclarePairedDelimiter\ceil{\lceil}{\rceil}
% \DeclarePairedDelimiter\floor{\lfloor}{\rfloor}

\let\mathbbalt\mathbb  % unicode-math changes mathbb to mathbbalt by default % https://tex.stackexchange.com/questions/360607/unicode-math-but-ordinary-blackboard-bold/360637#360637

% ---------- 'increasing spacing between matrix rows' -------------------- %
% https://tex.stackexchange.com/questions/14071/how-can-i-increase-the-line-spacing-in-a-matrix

% a redefinition of an internal amsmath LaTeX macro for customizing line spacing
% in  specific matrices  arbitrarily  as  desired: After  putting  this in  your
% preamble, you can write \begin{pmatrix}[1.5] vary  the value as you like, with
% pmatrix, vmatrix, bmatrix  and alike, or use it without  the optional argument
% as usually.

% \makeatletter
% \renewcommand*\env@matrix[1][\arraystretch]{%
%   \edef\arraystretch{#1}%
%   \hskip -\arraycolsep
%   \let\@ifnextchar\new@ifnextchar
% \array{*\c@MaxMatrixCols c}}
% \makeatother
  % some useful macros and settings for math
           % All math-related packages

% \usepackage{unicode-math} % which loads 'fontspec' (must be loaded only after graphicx and babel); https://tex.stackexchange.com/questions/188222/problem-with-babel-and-fontspec; no-math option (math-handling is left to unicode-math); silent to suppress all warnings (even in log file)
% %!TEX root = ../main.tex
% vim:nospell
% chktex-file 26

\setmainfont{LibertinusSerif}[%
	Extension         = .otf,
	% Path              = ./fonts/,
	UprightFont       = *-Regular,
	BoldFont          = *-Bold,
	ItalicFont        = *-Italic,
	BoldItalicFont    = *-BoldItalic,
	Numbers           = {Proportional, Lining},
	Ligatures         = {TeX, Common, Required},
	SmallCapsFeatures = {Letters = SmallCaps},
	% Fractions         = On,
	Kerning           = {Uppercase, On},
	% FontFace       = {sb}{n}{*-Semibold},
	% FontFace       = {sb}{it}{*-SemiboldItalic},
]%

\setsansfont{LibertinusSans}[%
	Extension         = .otf,
	% Path              = ./fonts/,
	UprightFont       = *-Regular,
	BoldFont          = *-Bold,
	ItalicFont        = *-Italic,
	Numbers           = {Proportional, Lining},
	Ligatures         = {TeX, Common, Required},
	SmallCapsFeatures = {Letters = SmallCaps},
	% Fractions         = On,
	Kerning           = {Uppercase, On},
]%

% \setmainfont[Numbers={Proportional},Ligatures={TeX, Common%, Historic, Contextual, Rare, Discretionary
% }]{Libertinus Serif}
% \setsansfont{Libertinus Sans}

\setmonofont{Latin Modern Mono}

% % https://tex.stackexchange.com/questions/103379/minionpro-semibold-medium
% \DeclareRobustCommand\sbseries{\fontseries{sb}\selectfont}
% \DeclareTextFontCommand{\textsb}{\sbseries}

\defaultfontfeatures{ Scale = MatchLowercase }
\defaultfontfeatures[\rmfamily]{ Scale = 1}

% \setmathfont[bold-style = ISO]{LibertinusMath-Regular.otf} % https://github.com/libertinus-fonts/libertinus/issues/20
\setmathfont{LibertinusMath-Regular}[%
	Extension  = .otf,
	% Path       = ./fonts/,
	bold-style = ISO, % https://github.com/libertinus-fonts/libertinus/issues/20
]%

% \setmathfont{libertinusmath-bold}[%
% Extension  = .otf,
% Path       = ./fonts/,
% bold-style = ISO, % https://github.com/libertinus-fonts/libertinus/issues/20
% version    = bold,
% ]%

% \setmathfont[math-style=upright,range={`e,`i}]{Latin Modern Math}
% \setmathfont[range={\mathunder,\triangleq},Scale=MatchUppercase]{STIX2Math.otf}
% \setmathfont[range = {\mathunder,\triangleq,\underbrace},Scale = MatchUppercase]{Latin Modern Math}
 % choose text, math fonts & monospaced fonts here

% \usepackage{fontsetup}    % Easier front-end to the fontspec package for use with unicode fonts
\usepackage{libertinus}

\usepackage{ragged2e}     % 'ragged2e' should be loaded after the body font & size have been established

\usepackage{microtype}    % for 'microtype' if using the option 'babel=true', babel must be loaded before microtype; microtype must be loaded after font selection (i.e. after fontspec/unicode-math)
%!TEX root = ../main.tex
% vim:nospell

% http://www.khirevich.com/latex/microtype/
\SetTracking{encoding={*}, shape=sc}{40} % default spacing is too loose, at least for IEEE style typography. To avoid increased inter-character spacing, a possible solution is to use small tracking value of 40 (which is 0.04 of 1 em)


% \usepackage{selnolig}     % luatex package; load after babel

\usepackage{csquotes}     % The fvextra package is loaded by 'minted', so you should load 'minted' before 'csquotes'; 'csquotes' must be loaded after biblatex
%!TEX root = ../main.tex
% vim:nospell ft=tex

\DeclareQuoteStyle[british]{english}
{\itshape\textquotedblleft} % chktex 6
[\textquotedblleft]
{\textquotedblright}
[0.05em]
{\textquoteleft}
{\textquoteright}


\usepackage{listings}
%!TEX root = ../main.tex
% vim: nospell

\usepackage{totpages}   % count pages in a document, and report last page number
%!TEX root = ../main.tex
% vim:nospell

\AtBeginDocument{%
    \hypersetup{%
    pdfnumpages={\ref*{TotPages}},%
    pdfpagerange={1-\ref*{TotPages}}}%
}


\usepackage{catchfile}  % catch an external file into a macro
%!TEX root = ../main.tex
% vim:nospell

\CatchFileDef{\ManuscriptKeywords}{frontmatter/keywords.tex}{}
% \CatchFileDef{\ManuscriptAuthors}{frontmatter/authors.tex}{}
\CatchFileDef{\ManuscriptTitle}{frontmatter/title.tex}{}
\CatchFileDef{\PageHeaderLeft}{frontmatter/pageheader_left.tex}{}
\CatchFileDef{\PageHeaderRight}{frontmatter/pageheader_right.tex}{}


\usepackage{trimspaces}
%!TEX root = ../main.tex
% vim:nospell
% chktex-file 1

\makeatletter
\trim@spaces@in\ManuscriptKeywords
% \trim@spaces@in\ManuscriptAuthors
\trim@spaces@in\ManuscriptTitle
\trim@spaces@in\PageHeaderLeft
\trim@spaces@in\PageHeaderRight
\makeatother


\usepackage{hyperxmp} % comment out if using the pdfx package
\usepackage{hyperref} % comment out if using the pdfx package

\hypersetup{%
	pdfauthor          = {Christopher Edsall, Kacper Kornet, Stefanie Reuter, Thomas Kappas, Tuomas Koskela, Krishnakumar Gopalakrishnan, Alexander Wade, Ioannis Zacharoudiou, Jon McCullough, Shunzhou Wan, James Richings},
	pdfcontactaddress  = {DiRAC project office, Science and Technology Facilities Council, UK},
	pdfcontactcity     = {Edinburgh},
	pdfcontactcountry  = {United Kingdom},
	pdfcontactemail    = {cje57 [at] cam [dot] ac [dot] uk},
	pdfcontactregion   = {Cambridge},
	pdfcontacturl      = {http://dirac.ac.uk, dirac-support@epcc.ed.ac.uk, mailto:c.jenner@ucl.ac.uk},
	pdfcreator         = {LuaTeX with hyperref},
	pdfkeywords        = {\ManuscriptKeywords},
	pdflang            = {en-GB},
	pdfsource          = {},  % a rarely needed option, this overrides the name of the LATEX source file.pdfsourceIt defaults to\jobname.texbut can be replaced by any other string.  If pdfsource is given an empty argument, no document source will be specified at all.
	pdfsubject         = {\ManuscriptKeywords},
	pdfsubtitle        = {DiRAC Federation Project WP 3.2.1},
	pdftitle           = {\ManuscriptTitle},
	pdftype            = {Technical report},
	pdfversionid       = {1.0},
	% pdfversionid       = {\IfPackageLoaded{gitver}{\gitVer}{\documentversion}},
	anchorcolor        = black,
	breaklinks         = true,
	citecolor          = black, %[RGB]{0,68,136}, % https://personal.sron.nl/pault/#fig:scheme_highcontrastpault/#fig:scheme_highcontrast
	colorlinks         = true,
	final              = true,
	linkcolor          = black, %[RGB]{0,68,136}, % https://personal.sron.nl/pault/#fig:scheme_highcontrastpault/#fig:scheme_highcontrast
	linktocpage        = true,
	pdfborderstyle     = {/S/U/W 1},
	pdfcenterwindow    = true,
	pdfdisplaydoctitle = true,
	pdffitwindow       = true,
	pdfstartview       = {Fit},
	pdftoolbar         = true,
	plainpages         = false,
	pdfencoding        = auto,
	unicode            = true,
	urlcolor           = black, %[RGB]{0,68,136}, % https://personal.sron.nl/https://personal.sron.nl/pault/#fig:scheme_highcontrastpault/#fig:scheme_highcontrast
}%


% ---------- packages to be loaded after hyperref ----------%

\usepackage{hypcap}      % to be loaded after hyperref. fix hyperref links to jump directly to table or figure

\usepackage{pdftexcmds}  % not required if using pdfx package

% \usepackage{glossaries-extra} % should be loaded after hyperref, but before cleveref % consider 'symbols' option

\usepackage{hypdestopt}  % seems to have problems with pdfx package?
\usepackage{bookmark}    % improves bookmarks handling. % More features & faster updated bookmarks.
\usepackage{cleveref}

% \usepackage{showframe}
% \usepackage[noframe]{showframe}
\newcommand{\blackurl}{\hypersetup{urlcolor=black}}%
\newcommand{\regularurl}{\hypersetup{urlcolor=[RGB]{0,68,136}}}%

% https://tex.stackexchange.com/a/330980
\pdfstringdefDisableCommands{%
\def\\{}%
\def\texttt#1{<#1>}%
}
  % hyperxmp, hyperref, nameref, algorithm, hypcap, bookmark, glossaries, cleveref, showframe etc.

% \usepackage{orcidlink}    % loads hyperref and tikz
\usepackage{fontawesome5}
%!TEX root = ../main.tex
% vim:nospell

% https://tex.stackexchange.com/a/627655

% Depends on the 'xcolor' package
\definecolor{orcidlogocol}{rgb}{0.65, 0.807, 0.223}
\newcommand{\orcidlink}[1]{$\,$\href{https://orcid.org/#1}{\textcolor{orcidlogocol}{\faOrcid}}}


% \usepackage{impnattypo}   % luatex package
\usepackage{nowidow}

\usepackage{xurl}         % to be loaded after biblatex if biblatex interaction is needed
%!TEX root = ../main.tex
% vim:nospell

% https://tex.stackexchange.com/a/612685
% \urldef\copernicus\url{https://cds.climate.copernicus.eu/cdsapp#!/dataset/reanalysis-era5-pressure-levels?tab=overview}

\urldef\oneapiaptinstall\url{https://www.intel.com/content/www/us/en/develop/documentation/installation-guide-for-intel-oneapi-toolkits-linux/top/installation/install-using-package-managers/apt.html#apt}
\urldef\hipsyclinstallfromrepos\url{https://github.com/illuhad/hipSYCL/blob/develop/install/scripts/README.md#installing-from-repositories}


%---------- end of package loading ---------%


%---------- begin custom commands ----------%

%!TEX root = ../main.tex
% vim:nospell

% \addbibresource{references.bib}
\addbibresource{zotero.bib}

\DeclareSourcemap{
	\maps[datatype=bibtex]{
		\map{
			\step[fieldsource=doi,final]
			\step[fieldset=url,null]
		}
	}
}

\DeclareSourcemap{
	\maps[datatype=bibtex]{
		\map[overwrite]{
			\step[fieldsource=month, match=\regexp{\Ajan\Z}, replace=1]
			\step[fieldsource=month, match=\regexp{\Afeb\Z}, replace=2]
			\step[fieldsource=month, match=\regexp{\Amar\Z}, replace=3]
			\step[fieldsource=month, match=\regexp{\Aapr\Z}, replace=4]
			\step[fieldsource=month, match=\regexp{\Amay\Z}, replace=5]
			\step[fieldsource=month, match=\regexp{\Ajun\Z}, replace=6]
			\step[fieldsource=month, match=\regexp{\Ajul\Z}, replace=7]
			\step[fieldsource=month, match=\regexp{\Aaug\Z}, replace=8]
			\step[fieldsource=month, match=\regexp{\Asep\Z}, replace=9]
			\step[fieldsource=month, match=\regexp{\Aoct\Z}, replace=10]
			\step[fieldsource=month, match=\regexp{\Anov\Z}, replace=11]
			\step[fieldsource=month, match=\regexp{\Adec\Z}, replace=12]
		}
	}
}


% https://tex.stackexchange.com/questions/113039/trying-to-suppress-urls-with-biblatex-using-a-simple-persons-method
% \AtEveryCitekey{\clearfield{url}}

% https://tex.stackexchange.com/questions/46787/is-there-a-way-to-prevent-urls-from-appearing-in-biblatex-citations
\AtEveryCitekey{%
	\clearfield{url}%
	\clearfield{urlyear}%
	\clearfield{doi}%
}%
\renewbibmacro*{in:}{}


% https://tex.stackexchange.com/questions/187684/abbreviate-only-middle-names-in-biblatex?rq=1
\makeatletter
\def\@empty{}
\def\first#1{\expandafter\@first#1 \@nil}
\def\@first#1 #2\@nil{#1\addspace%
	\if\relax\detokenize{#2}\relax\else\@initials#2\@nil\fi}
\def\initials#1{\expandafter\@initials#1 \@nil}
\def\@initials#1 #2\@nil{%
	\initial{#1}%
	\def\NextName{#2}%
	\ifx\@empty\NextName\relax%
	\else\bibinitdelim \@initials#2\@nil\fi}  % chktex 1
\def\initial#1{\expandafter\@initial#1\@nil}
\def\@initial#1#2\@nil{#1\bibinitperiod}
\makeatother


\DeclareBibliographyAlias{software}{online}

% \AtBeginBibliography{\vspace*{-10mm}}
% \renewcommand*{\bibfont}{\small} % helps towards maing bibliography left aligned, not justified

% https://tex.stackexchange.com/questions/264084/remove-url-from-bibliography-but-not-the-actual-url
\DeclareFieldFormat{url}{\url{#1}}
          % the 'bib' file goes in here
%!TEX root = ../main.tex
% vim:nospell ft=tex

\renewcommand{\arraystretch}{1.3}

% \newcolumntype{P}[1]{>{\RaggedRight\hspace{0pt}}p{#1}}
% % \newcolumntype{R}[1]{>{\raggedleft\let\newline\\\arraybackslash\hspace{0pt}}m{#1}}
% \newcolumntype{C}[1]{>{\centering\arraybackslash}p{#1}}

% \makeatletter
%
% \newcommand*{\@rowstyle}{}
%
% \newcommand*{\rowstyle}[1]{% sets the style of the next row
%     \gdef\@rowstyle{#1}%
%     \@rowstyle\ignorespaces%
% }
%
% \newcolumntype{=}{% resets the row style
%     >{\gdef\@rowstyle{}}%
% }
%
% \newcolumntype{+}{% adds the current row style to the next column
%     >{\@rowstyle}%
% }
%
% \makeatother



% \makeatletter
% \newlength{\qrr@dimen@}
% \expandafter\pretocmd\csname tabular*\endcsname{\setlength{\qrr@dimen@}{#1}}{}{}
% \newcommand*{\Rowcolor}[2][\tabcolsep]{%
%     \ifx\relax#1\relax\else
%         \kern-\the\dimexpr#1\relax
%     \fi
%     \makebox[0pt][l]{%
%         \fboxsep=0pt
%         \colorbox{#2}{%
%             \strut\kern\qrr@dimen@
%         }%
%     }%
%     \ifx\relax#1\relax\else
%         \kern\the\dimexpr#1\relax
%     \fi
%     \ignorespaces
% }
% \makeatother

%!TEX root = ../main.tex
% vim:nospell ft=tex
% chktex-file 1

% these custom commands  are general purpose definitions that are  suitable in a
% typical scientific	document. Some	of them  are pure  latex while	others use
% external packages

\newcommand\blankpage{
	\null
	\thispagestyle{empty}
	\addtocounter{page}{-1}
	\newpage
}

%---------- for text and other typographical elements ----------%
\newcommand{\eg}{\textit{e}.\textit{g}.}
\newcommand{\etal}{\textit{et~al}.}
\newcommand{\ie}{\textit{i}.\textit{e}.,}
\newcommand{\viz}{\textit{viz}. }

\setlength\parskip{0.75\baselineskip plus0.1\baselineskip  minus0.1\baselineskip}

% https://tex.stackexchange.com/questions/23487/how-can-i-get-roman-numerals-in-text
% \makeatletter
% \newcommand*{\romanletter}[1]{\expandafter\@slowromancap\romannumeral #1@}
% \makeatother

% improved handling of sectioning commands with titlesec package
% \setcounter{secnumdepth}{3} % organisational level that receives a numbers
% \setcounter{tocdepth}{3}		% print table of contents for level 3

% \setlength{\columnsep}{20pt} % space between columns in two column mode; default 10pt quite narrow

\renewcommand{\footnoterule}{%
	\kern -3pt
	\hrule width 0.25\textwidth height 0.5pt
	\kern 2pt
}

% % https://tex.stackexchange.com/questions/29916/how-to-place-a-footnote-inside-a-float-environment
% \newcommand{\mpfootnotes}[1][1]{
%			\renewcommand{\thempfootnote}{\thefootnote}
%			\addtocounter{footnote}{-#1}
% \renewcommand{\footnote}{\stepcounter{footnote}\footnotetext[\value{footnote}]}}

%!TEX root = ../main.tex
% vim:nospell ft=tex

\renewcommand{\topfraction}{.85}
\renewcommand{\bottomfraction}{.7}
\renewcommand{\textfraction}{.15}
\renewcommand{\floatpagefraction}{.66}
\renewcommand{\dbltopfraction}{.66}
\renewcommand{\dblfloatpagefraction}{.66}
\setcounter{topnumber}{9}
\setcounter{bottomnumber}{9}
\setcounter{totalnumber}{20}
\setcounter{dbltopnumber}{9}

% https://tex.stackexchange.com/questions/50830/do-i-have-to-care-about-bad-boxes/50850#50850
\tolerance=1414
\hbadness=1414
\emergencystretch=1.5em
\hfuzz=0.5pt
\vfuzz=\hfuzz  % chktex 1
\raggedbottom  % chktex 1
\hyphenpenalty=750
\frenchspacing
\binoppenalty=1000 % default 700
\relpenalty=800     % default 500
\interfootnotelinepenalty=10000
% \clubpenalty=10000
% \widowpenalty=10000

% \overfullrule=2cm
             % adjusts penalites & other settings for tex's page layout algorithm

% %!TEX root = ../main.tex
% vim:nospell ft=tex

%% Define custom colors to be used in the document here
% Load the xcolor package before using these

% \definecolor{brickreddvipsnames}{RGB}{173,51,38}
% \definecolor{cbrewer8classgreys6}{RGB}{115,115,115}
% \definecolor{cbrewerdarkblue}{RGB}{49,130,189}
% \definecolor{cbrewerdarkgray}{RGB}{99,99,99}
% \definecolor{cbrewerintergray}{RGB}{189,189,189}
% \definecolor{cbrewerlightgray}{RGB}{240,240,240}
% \definecolor{cbreweryellow2}{RGB}{255,247,188}
% \definecolor{cooldarkgrey}{RGB}{128,128,128}
% \definecolor{cornsilk}{RGB}{255,248,220}
% \definecolor{floralwhite}{RGB}{255,250,240}
% \definecolor{gnbu1}{RGB}{240,249,232}
% \definecolor{gnbu2}{RGB}{186,228,188}
% \definecolor{gnbu3}{RGB}{123,204,196}
% \definecolor{gnbu4}{RGB}{67,162,202}
% \definecolor{gnbu5}{RGB}{8,104,172}
% \definecolor{imperialblue}{RGB}{0,62,116}
% \definecolor{imperialbrick}{RGB}{165,25,0}
% \definecolor{imperialcoolgray}{RGB}{157,157,157}
% \definecolor{imperialdarkgreen}{RGB}{2,137,59}
% \definecolor{imperiallightblue}{RGB}{0,110,175}
% \definecolor{imperiallightgray}{RGB}{235,238,238}
% \definecolor{imperialnavy}{RGB}{0,33,71}
% \definecolor{imperialnewblue}{RGB}{0,86,146}
% \definecolor{imperialprocessblue}{RGB}{0,133,202}
% \definecolor{imperialraspberry}{RGB}{145,0,72}
% \definecolor{intercoolgrey}{RGB}{173,173,173}
% \definecolor{intermediategray}{RGB}{196,196,196}
% \definecolor{ivory}{RGB}{255,255,240}
% \definecolor{lightintergray}{RGB}{215,217,217}
% \definecolor{mahoganydvipsnames}{RGB}{161,53,40}
% \definecolor{mintedbg}{rgb}{0.95,0.95,0.95}
% \definecolor{mourablue}{RGB}{178,239,255}
% \definecolor{mouraorange}{RGB}{255,205,160}
% \definecolor{oldlace}{RGB}{253,245,230}
% \definecolor{seashell}{RGB}{255,245,238}
% \definecolor{sepiadvipsnames}{RGB}{99,29,11}
% \definecolor{viridistendarkblue}{RGB}{56,88,140}
% \definecolor{viridistenlighterblue}{RGB}{45,111,142}
% \definecolor{viridistwentybluefive}{RGB}{60,78,138}
% \definecolor{viridistwentyblueseven}{RGB}{49,103,142}
% \definecolor{viridistwentybluesix}{RGB}{54,91,141}

% %!TEX root = ../main.tex
% vim:nospell ft=tex

\WarningFilter{latex}{Marginpar on page}
% \WarningFilter{scrreprt}{Usage of package `titlesec'}
% \WarningFilter{scrreprt}{Activating an ugly workaround}
\WarningFilter{titlesec}{Non standard sectioning command detected}
\WarningFilter{microtype}{protrusion codes list}
\WarningFilter{latexfont}{Font}
\WarningFilter{latexfont}{Some font shapes}

% %!TEX root = ../main.tex
% vim:nospell ft=tex

\DeclareCaptionLabelSeparator{note}{\footnotemark \hspace*{0.5em}} % chktex 1

% %!TEX root = ../main.tex
% vim:nospell ft=tex

%%%%% Cross-reference formatting customised to IEEE style %%%%%%%

% \Crefname{type}{singular}{plural}
\crefname{equation}{}{}
\crefname{figure}{Fig.}{Fig.}

% \crefrangeformat{equation}{eqs.~(#3#1#4)--(#5#2#6)}

\newcommand{\crefrangeconjunction}{--}
% \crefrangeformat{equation}{eqs.~(#3#1#4)--(#5#2#6)}

% https://tex.stackexchange.com/questions/235516/cleveref-and-pdf-bookmark
\pdfstringdefDisableCommands{\let\Cref\autoref}

% https://tex.stackexchange.com/questions/251491/math-symbol-in-section-heading
\pdfstringdefDisableCommands{\def\varepsilon{\textepsilon}}

% https://tex.stackexchange.com/questions/193947/texorpdfstring-for-a-full-book
\pdfstringdefDisableCommands{\let\ensuremath\@gobble}

% %!TEX root = ../../main.tex
% vim:nospell ft=tex

% \makeglossaries  % IEEE TSTC doesn't explictly require a nomenclature list
% \preto\section{\glsresetall} % expand acronyms every chapter https://tex.stackexchange.com/questions/435617/glossaries-expand-acronyms-for-first-time-use-within-each-chapter/435680#435680
\glssetcategoryattribute{acronym}{nohyperfirst}{true} % no hyperlink on first use for entries with category=acronym % https://tex.stackexchange.com/questions/434160/line-break-long-glossaries-entry-when-using-hyperref-and-latex-dvips-ps2pdf
\setabbreviationstyle[acronym]{long-short} % applicable only for glossaries-extra.sty
% always set the abbreviation style before \GlsXtrLoadResources

\GlsXtrLoadResources[
src={preamble/glossary_definitions/acronyms.bib},%
charset={utf8},
sort={en-US},% sort according to this locale
selection={all},% select all entries in the .bib files
type=acronym,% put these entries in the 'acronym' glossary
set-widest,% needed for 'alttree' styles
save-locations=false,% don't save locations
replicate-fields={short=name} % https://tex.stackexchange.com/questions/481948/list-of-acronyms-alignment-error-using-bib2gls
]

\GlsXtrLoadResources[
src={preamble/glossary_definitions/latin_symbols.bib},%
charset={utf8},
sort={letter-lowerupper},
type=symbols,
selection={all},
category={same as entry},% set the category to the entry type
group={latin},% assign group label
set-widest,% needed for 'alttree' styles
save-locations=false
]

\newglossary*{greeksymbols}{greeksymbols}
\GlsXtrLoadResources[
src={preamble/glossary_definitions/greek_symbols.bib},%
charset={utf8},
sort={letter-lowerupper},
type=greeksymbols,% put these entries in the appropriate glossary
selection={all},
category={same as entry},% set the category to the entry type
group={greek},% assign group label
set-widest,% needed for 'alttree' styles
save-locations=false% don't save locations
]

% \GlsXtrLoadResources[
% src={preamble/glossary_definitions/generic_symbols.bib},%
% charset={utf8},
% sort={none},
% type=symbols,
% selection={all},
% category={same as entry},% set the category to the entry type
% group={generic},% assign group label
% set-widest,% needed for 'alttree' styles
% save-locations=false
% ]
% % \input{preamble/glossary_definitions/acronyms.tex} % load the acronym entries
% % \loadglsentries{preamble/acronyms}

% assign titles to group labels:
\glsxtrsetgrouptitle{latin}{\textnormal{\itshape A. Roman}} % chktex 1, chktex 6
\glsxtrsetgrouptitle{greek}{\textnormal{\itshape B. Greek}} % chktex 1, chktex 6
\glsxtrsetgrouptitle{generic}{\textnormal{\itshape C. Subscripts}} % chktex 1, chktex 6

% \glssetwidest{consectetuer}
% https://www.dickimaw-books.com/gallery/glossaries-styles/
\glsfindwidesttoplevelname  % chktex 1


%%%%%%%%%%%%%%%%%%%%%%%%%%%%%%%%%%%%%%%%%%%%%%%%%%%%%%%%%%%%%%%%%%%%%%%%%%%%%%%%%%%%%%%%%%
% \setglossarypreamble[acronym]{\small} % IEEE TSTC (optional) nomenclature font size
% \renewcommand*{\glstextformat}[1]{\textcolor{black}{#1}} % link coloring to match normal text, ie black. Has been set using hyperref custom commands

% Random trials below%%%%%%%%%%%%%%%%%%%%%%%%%%%%%%%%%%%%%%%%%%%%%%%%%%%%%%%%%%%%%%%%%%%%%%%%%%%%%%%%
% \renewcommand*{\glossaryheader}{}%
% \renewcommand*{\glsgroupheading}[1]{\textnormal{\glsgetgrouptitle{##1}}}%
% \renewcommand*{\glsgroupheading}[1]{\emph{#1}}
% \glssetcategoryattribute{symbols}{glossdescfont}{emph}
% \glssetcategoryattribute{symbols}{glossnamefont}{emph}
% \glssetcategoryattribute{symbol}{glossdescfont}{emph}
% \glssetcategoryattribute{symbol}{glossnamefont}{emph}
% % \renewcommand{\glsnamefont}[1]{\textsf{#1}} %Change acronym name font
% \renewcommand{\glsnamefont}[1]{\textsf{#1}} %Change acronym name font

% %!TEX root = ../main.tex
% vim:nospell ft=tex

\setlength{\ULdepth}{4.0pt}
\renewcommand{\ULthickness}{0.75pt}


%---------- end custom commands ----------%


\setstretch{1.348361657291667}
