% chktex-file 1
% chktex-file 44
% chktex-file 13

\documentclass[../main]{subfiles}

\begin{document}

\subsection{OpenQCD}\label{sec:openqcd}

OpenQCD-FASTSUM\footnote{\url{https://gitlab.com/fastsum/openqcd-fastsum/}} is a code for Lattice QCD simulations.
The main development is in C, with some functions heavily optimized using intrinsics from various instruction sets.
This work builds upon previous DiRAC work, where the code was profiled and the main hotspot was identified as the Dirac-Wilkinson (DW) operator.
The DW operator was ported to CUDA and a data layout optimization was performed to improve performance on V100 GPUs.

\subsubsection{Porting approach}\label{sec:openqcd_porting}

Since the DW operator was identified as the main hotspot in the code and a CUDA port had already been made, our approach was to follow Intel's Recommended Workflow illustrated in Figure~\ref{fig:intel-workflow}, starting from existing CUDA code, and use the Intel DPC++ Compatibility Tool (dpct) to generate DPC++ code from it.

We performed the porting on CSD3.
We found it difficult to work on the Intel DevCloud, since it did not have a CUDA library installed and dpct needs to access the CUDA header files.
Intel can't provide CUDA headers because of copyright issues. However, they do have a tutorial on how to use \texttt{dpct} on the DevCloud\footnote{\url{https://www.intel.com/content/www/us/en/developer/articles/training/intel-dpcpp-compatibility-tool-training.html}}.
As a short summary, the user has to provide the CUDA header files. The tutorial gives instructions on where to place them.

On CSD3 our workflow for dpct was the following:

\begin{enumerate}
	\item Set up modules
	      \begin{enumerate}
		      \item\texttt{module load cuda}
		      \item\texttt{module load gcc}
	      \end{enumerate}
	\item Set up OneAPI environment
	      \begin{enumerate}
		      \item\texttt{source /usr/local/software/intel/oneapi/2022.1/setvars.sh}
	      \end{enumerate}
	\item Run dpct
	      \begin{enumerate}
		      \item\texttt{cd <build directory>}
		      \item\texttt{intercept-build make}
		      \item\texttt{dpct -p compile\_commands.json --gen-build-script}
	      \end{enumerate}
\end{enumerate}

Since the build system used a simple Makefile, we used the intercept-build tool capture the source files and compiler commands from it.
This simplified the use of dpct significantly.
The \verb !--gen-build-script! flag was used to automatically generate a new Makefile for the ported dpc++ code, although the generated Makefile required some manual editing afterwards.
We will describe that in more detail in Section~\ref{sec:openqcd_compiling}.

After running dpct we got a list of warnings where dpct suggests attention is required from the developers.
These are reasonably clear and additional information is available in the documentation.
The warnings are listed in a closed issue on the GitHub repository\footnote{\url{https://github.com/UCL/openqcd-oneapi/issues/20}}.
None of them required our immediate attention to compile and run the code.
From the point of portability, we note that dpct warns us when workgroup sizes are hardcoded for a particular GPU architecture and recommends querying the device for its maximum workgroup size.

\subsubsection{Namespaces}\label{sec:openqcd_namespaces}

The dpc++ code produced by dpct includes a namespace \texttt{dpct} from header files within the OneAPI package.
In our case this was limited to two functions \verb!dpct::atomic_fetch_add()! and \verb!{dpct::get_device()!.
In order to make the code portable to other SYCL implementations one would need to replace the references to the (OneAPI specific) \texttt{dpct} namespace with standard \texttt{sycl} functions, or custom code.

The \verb!dpct::atomic_fetch_add()! function is included from the \texttt{atomics.hpp} header file.
The header files are distributed with a fairly permissive Apache-2.0 license with LLVM exceptions that allows to re-use them as long as all the modifications are noted.
Our solution was to include the \verb!dpct::atomic_fetch_add()! function within our source code.
It is a fairly thin wrapper that calls to the \texttt{sycl} namespace.

The \verb!dpct::get_current_device()! function returns a handle to the current device of type \verb #dpct::device_ext#. In the generated code, this handle is then used to get the default queue object using the \verb #default_queue()# method. It is not clear why this custom approach has been used by \texttt{dpct}. The default queue of the current device can be retrieved by standard \texttt{sycl} calls using

\begin{verbatim}
  sycl::queue q_ct1{ sycl::default_selector{} }
\end{verbatim}

which is what used to replace the \texttt{dpct} code\footnote{\url{https://github.com/UCL/openqcd-oneapi/pull/23}}. As an additional benefit, the \texttt{dpct} generated code was retrieving the current device and default queue in each kernel that was allocating or freeing memory on the device, creating unnecessary overhead. Using standard \texttt{sycl} namespace calls, we were able to create a queue once in the main function, and pass a reference to it to all kernels.

\subsubsection{Compiling}\label{sec:openqcd_compiling}

\paragraph{OneAPI}

To compile the code with OneAPI targeting Intel CPUs, we used the \texttt{dpcpp} compiler included in the OneAPI 2022.1 package.
This worked out of the box using the automatically generated Makefile simply using the \texttt{make} command.

To compile the code with OneAPI targeting NVidia GPUs we had to use Intel's Open Source \verb #clang++# compiler\footnote{\url{https://github.com/intel/llvm}} installed in a private module (\verb #/usr/local/software/spack/spack-modules/dpcpp-cuda-20220220/linux-centos8-x86_64_v3/# ) on CSD3.

In addition to the compiler module, we had to load a recent \verb #gcc/11.2.0# module, and initialise the OneAPI environment, as mentioned in Section~\ref{sec:openqcd_porting}.
In addition to the usual Makefile options, one has to pass the flags \verb #-fsycl# and \verb #-fsycl-targets=nvptx64-nvidia-cuda# to compile sycl code targeting the NVidia CUDA backend.
The automatically generated Makefile did not work for the NVidia backend, and we manually modified it\footnote{\url{https://github.com/UCL/openqcd-oneapi/pull/42}} to simply contain a target of the type
\begin{verbatim}
my_target: $(SRC_FILES)
	$(CC) $(INCLUDES) $(CFLAGS) $^ -o $(EXEC_BASENAME).$@
\end{verbatim}
with the source files listed in \verb #$SRC_FILES#, the relevant compiler flags in \verb #$(CFLAGS)# and a path to the header files in \verb #$(INCLUDES)#.

\paragraph{hipSYCL}

As mentioned in section~\ref{sec:software_stack} we have hipSYCL available on CSD3 that can target NVidia A100 GPUs.
It is accessible through the module system with
\begin{verbatim}
module load hipsycl/0.9.2
\end{verbatim}.
Note that we required version 0.9.2, instead of 0.9.1 because support for atomic operations was added in 0.9.2.
Additionally for running a binary produced with hipSYCL it may be necessary to load a module providing newer libstdc++, for example:
\begin{verbatim}
module load gcc/9.4.0
\end{verbatim}
After the modifications to namespaces described in Section~\ref{sec:openqcd_namespaces}, we were able to use it to compile with CPU \texttt{omp} and GPU \texttt{cuda} backends.
One additional code modification was required, The hipSYCL implementation on wraps the \texttt{sycl} namespace inside the \texttt{cl} namespace.
We had to add \verb !using namespace cl;! to the dpcpp source file, this did not break compatibility with OneAPI.
To target Intel CPUs we added the \verb !--hipsycl-targets=omp! compiler flag, and to target NVidia A100 GPUs we added the \verb !--hipsycl-targets=cuda:sm_80! compiler flag.
Alternatively, one can modify the \verb !HIPSYCL_TARGETS! and \verb !HIPSYCL_GPU_ARCH! environment variables.

As further mentioned in section~\ref{sec:software_stack} it would be possible to install hipSYCL on Cosma8 to run on AMD GPUs. We made an effort to install it using the spack package manager.
Installing Spack is fairly straightforward, all that is required is cloning the GitHub repository\footnote{\url{https://github.com/spack/spack}} and ensuring python is installed.
This can be done in the user's home directory on a HPC system.
To be used effectively, spack requires an environment that points to system packages that should not be reinstalled by the user.
On Cosma8, we used an environment developed by the ExCALIBUR H\&ES project\footnote{\url{https://github.com/ukri-excalibur/excalibur-tests/blob/main/spack-environments/cosma8/compute-node/spack.yaml}}.
Once the environment is set up, we can install hipsycl with the system-provided version of the \texttt{boost} library and \texttt{gcc} compiler by
\begin{verbatim}
spack install hipsycl ^boost@1.67.0 %gcc@10.2.0
\end{verbatim}
The installed hipSYCL was successful in compiling for the CPU backend on Cosma8.
However, we discovered that the spack recipe did not support the ROCm backend\footnote{\url{https://github.com/UCL/openqcd-oneapi/issues/29}}, which is necessary in order to compile for AMD GPUs. The only viable way forward seems for the local sysadmin team to install hipSYCL from source.

\paragraph{ComputeCPP}

% \todo{@Krishna Add (issues with) compiling with ComputeCPP}

\subsubsection{Testing}\label{testing_openqcd}

The repository contains a stand-alone test\footnote{\url{https://gitlab.com/fastsum/openqcd-fastsum/-/tree/feature/cuda_tests/tests/cuda}} for the CUDA port that runs in less than a minute on most GPUs and verifies the results of the DW operator against reference data.
We found generating the reference data difficult using the main CPU code based on\footnote{\url{https://gitlab.com/fastsum/openqcd-fastsum/-/blob/feature/cuda_tests/tests/cuda/Dump\_memory\_guide.txt}}.
However, we were able to recover the original reference data and verify the CUDA code.
The files are temporarily stored in the project directory on CSD3 \texttt{/rds/project/dirac\_vol2/rds-dirac-dr004/openqcd/makis-ref-data}.
This is not a very sustainable solution however, and should be moved to, e.g. GitHub LFS or a similar data storage service. % \todo{@Krishna, Report on difficulties with Git LFS on GitHub}

\subsubsection{Continuous Integration}

It is possible to install prebuilt binaries of both OneAPI and hipSYCL compilers from repositories relatively quickly, which enables us to use CI workflows during the development work.
CI helps catch errors quickly, and ensure that code merged into the main development branch successfully builds and executes test cases.
It is standard practice in industry and we warmly recommend it for scientific software projects.
We chose to use GitHub Actions as the CI platform of our choice, due to good integration with the GitHub source repository and relatively friendly pricing rules, which allow us to run our CI jobs free of charge.
In GitHub Actions, the user defines a workflow in a \verb #.yaml# file which can consist of any number of arbitrary steps.
We implemented a simple workflow that runs the main executable on two data sets, using the OneAPI\footnote{\url{https://github.com/UCL/openqcd-oneapi/blob/e5d53c7e47d8b94f78c87b91ef218d217ce57597/.github/workflows/oneapi.yml}} and hipSYCL\footnote{\url{https://github.com/UCL/openqcd-oneapi/blob/e5d53c7e47d8b94f78c87b91ef218d217ce57597/.github/workflows/hipsycl.yml}} compilers, on a CPU-only machine running the latest Ubuntu version available.
It was not possible to test the GPU builds at this time.
To install the OneAPI compiler, we followed this blog\footnote{\url{https://neelravi.com/post/oneapi-github-workflow/}} and Intel's instructions\footnote{\oneapiaptinstall}.
Installing the whole OneAPI package would take too long for the CI job, the main issue was finding the right packages for installing the \verb #dpcpp# compiler only.
To install hipSYCL, we used instructions for installing from repositories from the developers\footnote{\hipsyclinstallfromrepos}.

\subsubsection{Performance}\label{sec:openqcd_performance}

To measure the performance on the NVidia A100 GPUs available on CSD3, we profiled the CUDA and the SYCL versions of the code with the NVidia NSight Compute profiler.
NVidia tools seemed the only feasible option for profiling performance on NVidia devices.

We found NSight compute works reasonably well on SYCL code.
The main issue is that since the kernels executed on the GPU are generated from lambda functions in SYCL, the function names are not retained as kernel names.
Instead, the profiled kernels have generic names that can be hard to decipher if the code contains multiple kernels.
To avoid this issue, it is recommended to name the kernels using a class passed as a template argument to \verb #parallel_for#, e.g.
\begin{verbatim}
    q.parallel_for<class name_of_my_kernel>(...)
\end{verbatim}
Note, that the names given to kernels have to be unique, i.e.\ not clash with names of functions or other code constructs.

Based on our initial comparisons, the CUDA kernels in OpenQCD perform better than the SYCL kernels we generated.
We have summarized the preliminary results in table~\ref{tab:openqcd_perf}.
The difference varies between the kernels we studied, from a factor of 2 to a factor of 6.
We note that the compute throughput seems categorically lower by roughly a factor of 2, while the memory throughput is comparable in the \verb #mulpauli# and \verb #doe# kernels and lower by a factor of 2 in the \verb #deo# kernel.
NSight Compute suggests the kernels are bottlenecked by memory throughput.
A significant difference between the memory use of the CUDA and SYCL kernels is the CUDA kernels are only using Global memory while SYCL kernels are using both Global and Local memory.
The roofline analysis gives an AI of 0.37 for the SYCL code and 1.15 for the CUDA code.
It seems safe to say that the SYCL code is performing memory accesses that may be unnecessary.
This requires further study and optimization.\todo{@Tuomas Add what we found. Occupancy vs register pressure seems to be the main indicator}

Changes to workgroup size did not have a significant effect on performance on NVidia A100s.

\begin{center}
	\begin{table}
		\begin{tabular}{@{}l l S S S S S@{}}
			\toprule
			\thead{Kernel      \\ \vphantom{(ms)}} & 
			\thead{Compiler    \\ \vphantom{(ms)}} & 
			{\thead{Duration   \\ (ms)}} & 
			{\thead{Compute    \\ throughput (\%)}} & 
			{\thead{Memory     \\ throughput (\%)}} & 
			{\thead{Occupancy \\ (\%)}} & 
			{\thead{Registers \\ per thread}} \\
			\midrule
			\texttt{mulpauli} & CUDA    & 4.85  & 16.05 & 81.28 & 60.09 & 48 \\
			\texttt{mulpauli} & OneAPI  & 17.21 & 7.39  & 70.98 & 23.30 & 116  \\
			\texttt{mulpauli} & hipSYCL & 8.12 & 10.34  & 78,51 & 17.60 & 168 \\
			\midrule
			\texttt{doe}      & CUDA    & 7.57  & 15.78 & 62.59 & 24.65 & 116 \\
			\texttt{doe}      & OneAPI  & 17.41 & 7.31  & 61.21 & 18.44 & 168 \\
			\texttt{doe}      & hipSYCL & 11.14 & 10.57  & 63.54 & 12.28 & 255 \\
			\midrule
			\texttt{deo}      & CUDA    & 8.24  & 12.94 & 54.93 & 23.92 & 108 \\
			\texttt{deo}      & OneAPI  & 46.32 & 7.64  & 27.22 & 18.54 & 142 \\
			\texttt{deo}      & hipSYCL & 54.90 & 4.80  & 30.96 & 18.60 & 168 \\
			\bottomrule
		\end{tabular}
		\caption{\label{tab:openqcd_perf}Performance comparison of CUDA and SYCL OpenQCD DW kernels on NVidia A100.}
	\end{table}
\end{center}

\todo{@Tuomas Update table with HIPSycl performance on NVidia A100s and add Occupancy and register use}

\subsubsection{Accessing Device and Host Memory}\label{sec:openqcd_memoryaccess}

% \todo{@Krishna Report on Explicit vs Implicit memory movement in USM, mention buffer/accessor model}

\subsubsection{Estimate of Development Effort}\label{sec:openqcd_personhours}

We had two developers (TK and KG) working on openqcd for most of the project duration.
This strategy helps us routinely use good software development practices, such as pair programming and code review.
The planned time allocation was 50~\% of time from both developers, however due to unforeseen personal circumstances we were not able to deliver all of the allocated time.
The starting point of the project was both developers were proficient c++ programmers, with varying knowledge of C, CUDA, and SYCL, but not familiar at all with the openqcd codebase.
As a very rough estimate, out of the six months spent on the project, the first two were spent getting familiar with the code, the Intel tools, and the programming languages.
The first meaningful commits to the repository were made two months after the start of the project\footnote{\url{https://github.com/UCL/openqcd-oneapi/commits/arc_dev}}, with fairly rapid development in the following month.
We spent a total of approximately 330 hours of development time within the six month period. This includes the development time, but also the time spent going through tutorials, meetings with the team, reading textbooks, planning and design.
In our experience, with the automatic translation tool \texttt{dpct}, it is quite straightforward to port pre-existing CUDA code into a SYCL code that will compile with OneAPI and run successfully on CPU and GPU hardware.
We estimate that most of the development effort went into making the code portable across different SYCL compilers, setting up the testing infrastructure and analysing the performance.

\end{document}
